\documentclass[12pt]{article}
\title{Interview Questions}
\author{Avneesh}


\begin{document}
\maketitle

StakeHolder Profile

\vspace{1cm}
\begin{tabular}{|p{3cm}|p{3cm}|}
\hline
Name & Hasan Ali \\
%\multicolumn{1}{|p{1.42in}}{\textbf{Name}} & 
\hline
University & Concordia \\
\hline
Industry & Education \\
\hline
\end{tabular}

\begin{enumerate}

 
\item What is your role in your company/education institute?

Ans:  Student in the mathematics department.

\item What is your area of research in the field of mathematics?

Ans:  Discrete mathematics

\item There are many constants that exists in the field of mathematics. Why choose Universal mathematical constant?

Ans:  The application is using parabola as its main component. 

\item	How will you calculate the value of Universal Parabolic Constant?

Ans:  The value is derived by using the below formula
       P= ln(1+  √2) + √2  

\item Since this number is irrational number. How can you ensure the accuracy of the results?
   
Ans:  Precision will be taken up to two decimal places.

\item	In which application, you are going to use this number?

Ans:  Calculation of solar heat consumed, and units of electricity produced.

\item Why you are developing this application?

Ans:  Since nonrenewable source of energy will eventually run out, there is a need to move           towards clean and renewable source of energy.

\item How frequently you are going to use this number?

Ans:  Very frequently

\item Can you tell me in which other applications, this number can be used?

Ans:  This number can be used in the application of Acoustic mirror and Reflecting Telescope


\item Can you explain me about the working of your application?

Ans:  The solar light will fall on the parabola surface and parabola will concentrate light on a            receiver material and retain heat. The retained heat will be used for cooking.

\item What is the future of this number?

Ans:  This number is a mathematical constant. And there are many researches going in the field  of parabola. It will assist in future projects related to the parabola.

\item Do you want to integrate or display the result of your application to a web application   for the usage to other people all over the world?

Ans:   Yes, I would display the results to my blog.

\item Is there any tool that exist to do the mathematical calculation for the application?

Ans:    No tool is being used. All the calculations are done manually.

\item Do you want to build any tool that can automate the process?

    Ans:     Yes, it would be helpful.

\end{enumerate}

\end{document}
